\appendix
\chapter{Appendix}

\section{Traffic Matrix Generation with Gravity Model Code for QUISP}
\begin{lstlisting}[language=c++,caption=Gravity Generator Function, label=gravity_cpp]
  void Application::gravityGenerator() {
    int traffic_ini = par("TrafficIni");
    int res;
    double res_all = 0;
  
    cTopology *topo = new cTopology("topo");
    topo->extractByParameter("nodeType", provider.getQNode()->par("nodeType").str().c_str());
  
    for (int i = 0; i < other_end_node_addresses.size(); i++) {
      cTopology::Node *node = topo->getNode(i);
      auto app_module = node->getModule()->getModuleByPath(".app");
      if (app_module == nullptr) throw cRuntimeError("app module not found");
      res = app_module->par("TrafficRes");
      res_all += res;
    }
    for (int i = 0; i < other_end_node_addresses.size(); i++) {
      cTopology::Node *node = topo->getNode(i);
      auto app_module = node->getModule()->getModuleByPath(".app");
      if (app_module == nullptr) throw cRuntimeError("app module not found");
      res = app_module->par("TrafficRes");
      traffic_matrix.push_back((int)round(traffic_ini * (res / res_all)));
    }
    int sum = std::accumulate(traffic_matrix.begin(), traffic_matrix.end(), 0);
    while (sum != traffic_ini) {
      std::vector<int>::iterator iter = std::max_element(traffic_matrix.begin(), traffic_matrix.end());
      size_t max_index = std::distance(traffic_matrix.begin(), iter);
      if (sum > traffic_ini) {
        traffic_matrix[max_index] -= 1;
      } else if (traffic_ini < sum) {
        traffic_matrix[max_index] += 1;
      }
      sum = std::accumulate(traffic_matrix.begin(), traffic_matrix.end(), 0);
    }
  
    delete topo;
  }
  \end{lstlisting}
  
  \begin{lstlisting}[language=c++,caption=inside of initialize function]
  if (traffic_pattern == 3) {
      gravityGenerator();
      for (int i = 0; i < other_end_node_addresses.size(); i++) {
          EV_INFO << traffic_matrix[i] << " requests will be sent from " << my_address << " to " << other_end_node_addresses[i] << "\n";
          printf("%d requests will be sent from %d to %d\n", traffic_matrix[i], my_address, other_end_node_addresses[i]);
      }
      return;
  }
  \end{lstlisting}
  

\section{Gravity Model Sample Code with Python}

\begin{lstlisting}[language=python,caption=Generate Simple Network with Python]
#Author: Nozomi Tanetani
import networkx as nx
import numpy as np
import matplotlib.pyplot as plt 

def graph_generator(n, p, seed):
  g = nx.random_graphs.fast_gnp_random_graph(n, p, seed, directed = True) #Generate a random graph
  g = g.to_undirected()

  #Initialize a number of traffic per node as 0.
  for i in g.nodes():
    g.nodes[i]['in'] = 0
    g.nodes[i]['out'] = 0

  #Store a random number into number of traffic variable.
  for i in g.nodes():
    g.nodes[i]['in'] = np.random.randint(n*100,n*500)
    tmp = g.nodes[i]['in']
    for j in g.nodes():
      if i is not j:
        if j is n-1:
          g.nodes[j]['out'] += tmp
        else:
          rand_tmp = np.random.randint(0,tmp)
          g.nodes[j]['out'] += rand_tmp
          tmp = tmp - rand_tmp
  return g

g = graph_generator(4, 1, 13)

#Display a generated graph
plt.figure(figsize=(8,8))

pos = nx.circular_layout(g)
nx.draw_networkx_nodes(g, pos, node_size=5500, node_color='gray')
nx.draw_networkx_edges(g, pos)
nx.draw_networkx_labels(g, pos, font_size=20)

#Display
labeldict = nx.get_node_attributes(g,'in')
for i in g.nodes():
  labeldict[i] = '\n\n' + str(labeldict[i])
nx.draw_networkx_labels(g,pos,labels=labeldict,font_size=12,font_color='#FFAAAA')

labeldict = nx.get_node_attributes(g,'out')
for i in g.nodes():
  labeldict[i] = '\n\n\n\n' + str(labeldict[i])
nx.draw_networkx_labels(g,pos,labels=labeldict,font_size=12,font_color='#AAFFFF')
print("red number is total traffic that enters the node.")
print("blue number is total traffic that exits the node.")
plt.show()
\end{lstlisting}

\begin{lstlisting}[language=python,caption=Gravity Generator Function with Python]
def gravity_generator(graph):
    tm = np.zeros((n,n))

    for i in range(len(tm)):
        for j in range(len(tm[i])):
        if i is not j:
            tmp = 0
            for k in range(n):
            if k is not i:
                tmp += graph.nodes[k]['out']
            tm[i][j] = int(np.round(graph.nodes()[i]['in'] * (graph.nodes[j]['out'] / tmp))) # Slide Page 15
    for i in range(len(tm)):
        if sum(tm[i]) - graph.nodes()[i]['in'] == 1:
        tm[i][np.argmax(tm[i])] -= 1
        elif graph.nodes()[i]['in'] - sum(tm[i]) == 1:
        tm[i][np.argmax(tm[i])] += 1
    return tm

#Create a traffic matrix
tm = gravity_generator(g)
print(tm)
\end{lstlisting}

\section{Unit Test Code}
Test code to confirm whether my traffic matrices generation calculation is correct.
\begin{lstlisting}[language=c++,caption=Test code for confirming the calculation, label=eval_code]
TEST(AppTest, Gravity_Model_3node) {
  auto *sim = prepareSimulation();
  auto *mock_qnode = new TestQNode{123};
  auto *mock_qnode2 = new TestQNode{456};
  auto *mock_qnode3 = new TestQNode{789};

  auto *app = new AppTestTarget{mock_qnode};
  setParBool(app, "EndToEndConnection", true);
  setParInt(app, "num_measure", 1);
  setParInt(app, "TrafficPattern", 3);
  setParInt(app, "distant_measure_count", 100);
  setParInt(app, "LoneInitiatorAddress", 123);
  setParInt(app, "TrafficIni", 90); //set the amount of initiator traffic
  setParInt(app, "TrafficRes", 0); //set the amount of responder traffic
  app->setName("app");
  call_private::insertSubmodule(*dynamic_cast<cModule *>(mock_qnode), app);
  sim->registerComponent(mock_qnode);
  sim->registerComponent(app);

  auto *app2 = new AppTestTarget{mock_qnode2};
  setParBool(app2, "EndToEndConnection", true);
  setParInt(app2, "num_measure", 1);
  setParInt(app2, "TrafficPattern", 3);
  setParInt(app2, "distant_measure_count", 100);
  setParInt(app2, "LoneInitiatorAddress", 123);
  setParInt(app2, "TrafficIni", 0); //set the amount of initiator traffic
  setParInt(app2, "TrafficRes", 60); //set the amount of responder traffic
  app2->setName("app");
  call_private::insertSubmodule(*dynamic_cast<cModule *>(mock_qnode2), app2);
  sim->registerComponent(mock_qnode2);
  sim->registerComponent(app2);

  auto *app3 = new AppTestTarget{mock_qnode3};
  setParBool(app3, "EndToEndConnection", true);
  setParInt(app3, "num_measure", 1);
  setParInt(app3, "TrafficPattern", 3);
  setParInt(app3, "distant_measure_count", 100);
  setParInt(app3, "LoneInitiatorAddress", 123);
  setParInt(app3, "TrafficIni", 0); //set the amount of initiator traffic
  setParInt(app3, "TrafficRes", 30); //set the amount of responder traffic
  app3->setName("app");
  call_private::insertSubmodule(*dynamic_cast<cModule *>(mock_qnode3), app3);
  sim->registerComponent(mock_qnode3);
  sim->registerComponent(app3);

  app->callInitialize();
  app2->callInitialize();
  app3->callInitialize();
  ASSERT_EQ(app->getAddress(), 123);
  ASSERT_EQ(app2->getAddress(), 456);
  ASSERT_EQ(app3->getAddress(), 789);

  ASSERT_EQ(app->gravityGenerator().size(), 2);
  ASSERT_EQ(app->gravityGenerator()[0], 60); //Assertion test wheather the amount of traffic from node 1 to node 2 is 60
  ASSERT_EQ(app->gravityGenerator()[1], 30); //Assertion test wheather the amount of traffic from node 1 to node 3 is 30
}
\end{lstlisting}